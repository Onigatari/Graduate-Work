\documentclass[a4paper,12pt]{article}
\usepackage[T1]{fontenc}
\usepackage[utf8]{inputenc}
\usepackage[russian]{babel}
\usepackage[14pt]{extsizes}
\usepackage{cmap}
\usepackage{indentfirst}
\usepackage{autonum}
\usepackage{amsfonts}
\usepackage{amsmath}
\usepackage{amssymb}
\usepackage{amsthm}
\usepackage{upgreek}
\usepackage{graphicx}
\usepackage{listings}
\usepackage{multirow}
\usepackage{dsfont}
\usepackage{setspace,amsmath}
\usepackage[table,xcdraw]{xcolor}
\usepackage[unicode, pdftex]{hyperref}
\usepackage[left=15mm, top=20mm, right=15mm, bottom=20mm, nohead, footskip=15mm]{geometry}
\usepackage{tikz}
\usepackage{tkz-graph}
\usepackage{listings}
\usepackage{color}

\definecolor{dkgreen}{rgb}{0,0.6,0}
\definecolor{gray}{rgb}{0.5,0.5,0.5}
\definecolor{mauve}{rgb}{0.58,0,0.82}

\begin{document}
	\selectlanguage{russian}
	\setcounter{page}{0}
	\renewcommand{\labelenumii}{\arabic{enumi}.\arabic{enumii}.}
	
	\lstset{
		frame=tb,
		framexleftmargin=1.5em,
		language=C++,
		aboveskip=2mm,
		belowskip=2mm,
		showstringspaces=false,
		extendedchars=\true
		inputencoding=utf8x,
		columns=flexible,
		basicstyle={\small\ttfamily},
		numbers=left,
		numberstyle=\tiny\color{gray},
		keywordstyle=\color{blue},
		commentstyle=\color{dkgreen},
		stringstyle=\color{mauve},
		breaklines=true,
		breakatwhitespace=true,
		tabsize=4
	}
	
	\begin{center}
		\small{Министерство науки и высшего образования Российской Федерации}\\
		\small{Федеральное государственное бюджетное образовательное учреждение}\\
		\small{Высшего образования}\\
		\small{\textbf{«Северо-Осетинский государственный университет\\
				имени Коста Левановича Хетагурова»}}\\
		
		\hfill \break
		\hfill \break
		\hfill \break
		\hfill \break
		\hfill \break
		\hfill \break
		\hfill \break
		\hfill \break
		\hfill \break
		
		\normalsize{Дипломная работа}\\
		\large{\textbf{Seq2seq подход для задач Машинного Перевода}}\\
		
		\hfill \break
		\hfill \break
		\hfill \break
		\hfill \break
		\hfill \break
		\hfill\break
	\end{center}
	
	\begin{flushright}
		\textbf{Выполнил:}\\
		Студент 4 курса направления:\\
		«Прикладная математика и информатика»\\
		\textit{Гамосов Станислав Станиславович \underline{\hspace{3cm}}}\\
	\end{flushright}
	
	\hfill
	
	\begin{flushright}
		\textbf{Научный руководитель:}\\
		Кандидат физико-математических наук:\\
		\textit{Басаева Елена Казбековна \underline{\hspace{3cm}}}\\
	\end{flushright}
	
	\hfill
	
	\begin{flushright}
		\textbf{Консультант}\\
		Старший преподаватель: \\
		\textit{Макаренко Мария Дмитриевна \underline{\hspace{3cm}}}\\
	\end{flushright}
	
	\normalsize{ \hspace{28pt}} \hfill \break
	\begin{center} Владикавказ 2022 \end{center}
	
	\thispagestyle{empty}
	\tableofcontents
	\thispagestyle{empty}
	\clearpage
	\newtheorem{theorem}{Теорема}
	
	\section{Введение}
	
	 \textbf{Seq2seq} - это целое семейство подходов машинного обучения, используемых для обработки естественного языка. Основными задачами для данной теории являются: языковой перевод, субтитры к изображениям, разговорные модели и обобщение текста.
	 
	 Алгоритм был разработан \textit{Google} для использования в машинном переводе. Как уже можно заметить за последнюю пару лет коммерческие системы стали удивительно хороши в  переводе - посмотрите, например, \textit{Google Translate}, \textit{Яндекс}-переводчик, переводчик \textit{DeepL}, переводчик \textit{Bing Microsoft}.
	 
	 Так же у \textbf{seq2seq} технологи огромный потанцевал, помимо привычного машинного перевода между естественными языками, вполне реализуем перевод между языками программирования (\textit{Facebook AI "Глубокое обучение переводу между языками программирования"}). Поэтому возможности применений такого подхода довольно велики. 
	 
	 Далее в своей работе под машинным переводом я будем понимать любую общую задачу \textbf{seq2seq}, если точнее, то перевод между последовательностями любой природы.
	 
	 \clearpage
	 
	 \section{Формализация задачи машинного перевода}
	 Формально в задаче машинного перевода у нас есть входная последовательность $x_{1}, x_{2}, ... x_{m}$ и последовательность вывода $y_{1}, y_{2}, ... y_{n}$, само собой длинна данных последовательностей может отличатся. Саму процедуру \textit{перевода} можно рассматривать как нахождение искомой последовательности, которая является наиболее вероятной с учетом входных данных. Формально искомая последовательность, которая максимизирует условную вероятность $p(y|x): y^{'} = argmax[p(y|x)]$.
	 
	 Когда человеку известны уже два языка с которыми он работает, то уже при переводе можно сказать насколько хорошо справилась модель, является ли перевод естественным и насколько он приятен на слух. Однако такой вид анализа неприемлем для машины, поэтому нам стоит проанализировать уже имеющуюся функцию $p(y|x,\theta)$ с неким параметром $\theta$, а затем найти его $argmax$ для $y^{'} = argmax_{y}[p(y|x, \theta)]$.
	 
	 Прежде чем перейти к самой задачи перевода, нужно ответить на 3 вопроса:
	 
	 \begin{itemize}
	 	\item \textbf{Моделирование}: Как работает модель для $p(y|x, \theta)$?
	 	\item \textbf{Обучение}: Как найти параметр $\theta$?
	 	\item \textbf{Вывод}: Как понять, что текущий $y$ лучший?
	 \end{itemize}
 
 	\clearpage
 	
 	\section{Структура Encoder-Decoder}
 	
 	Наиболее распространенная модель \textbf{Sequence-to-sequence (seq2seq}) являются модель \textbf{Encoder-Decoder}, в которой обычно используют \textbf{рекуррентную нейронную сеть} (\textbf{RNN}) для кодирования исходной последовательности в один вектор.
 	
 	На самом деле полученный вектор можно представить как набор образов сущностей с образами взаимоотношений между ними. Этот вектор затем декодируется вторым \textbf{RNN}, который учится выводить выходное предложение, генерируя его по одному слову за раз.
\end{document}